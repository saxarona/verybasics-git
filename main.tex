\documentclass[dvipsnames, usenames]{beamer}

\usepackage[english]{babel}
\usepackage[utf8]{inputenc}
\usepackage{graphicx}

\usetheme{Dresden}
\usecolortheme{whale}

\parskip=1.3ex
\newcommand{\spacepls}{\vspace{1.5Ex}}

\title{The very basics of Git}
\subtitle{Git for the non-programmer}

\author[X. Sánchez-Díaz]{Xavier~Sánchez-Díaz}
\institute[ITESM]{Center For Intelligent Computing and Robotics\\
Tecnológico de Monterrey, Campus Monterrey}
\date{\today}
\subject{The very basics of git}

\begin{document}

\begin{frame}
	\titlepage
\end{frame}

\begin{frame}[allowframebreaks]
	\frametitle{Outline}
	\tableofcontents
\end{frame}

\section{Preamble} % (fold)
\label{sec:preamble}

\subsection{The UNIX Philosophy} % (fold)
\label{sec:philosophy}

\begin{frame}[t]
	\frametitle{Everything is a file}
	\framesubtitle{The UNIX philosophy}

	\spacepls

	\begin{itemize}
		\item Source code is stored in files. \pause
		\item Word documents are files. \pause
		\item Photos are files. \pause
		\item Videos are files. \pause
		\item HDDs and USBs are files. \pause
	\end{itemize}

	\spacepls
	{\huge Everything \pause is \pause a \pause file.}\\
\end{frame}

\begin{frame}[t]
	\frametitle{Everything is a file}
	\framesubtitle{The UNIX philosophy}

	\spacepls

	Since everything is a file, it makes sense to keep track of the changes of your projects using git, no matter what you're working on!
	
\end{frame}

% subsection philosophy (end)

\subsection{What is git?} % (fold)
\label{sub:what_is_git}

\begin{frame}[t]
	\frametitle{What is Git?}
	\framesubtitle{And why should I use it?}

	\spacepls

	Git is a {\color{OliveGreen}free and open source} \alert{distributed} {\color{Blue}version control system}. \pause

	\begin{block}{Version control system}	
		A system that allows easy management project versions.
	\end{block} \pause

	\begin{alertblock}{Distributed}	
		Not centralized. Everyone on the team has their own local copy.
	\end{alertblock} \pause

	\begin{exampleblock}{Free and Open Source}	
		Git is distributed according to a GPLv2 License.
	\end{exampleblock}
	
\end{frame}


\begin{frame}[t]
	\frametitle{Free speech, not free beer}
	\framesubtitle{What is Git?}

	Git is free software since it respects the four essential freedoms of the user: \pause

	\begin{itemize}
		\item Freedom to use the program as you wish, for any purpose. \pause
		\item Freedom to modify the software as you want. The source code is there for you to look at and modify at will. \pause
		\item Freedom to share or redistribute copies as you wish to help your neighbor. \pause
		\item Freedom to share or redistribute copies of your modified versions to others.
	\end{itemize}
	
\end{frame}

\begin{frame}[t]
	\frametitle{What is Git?}
	\framesubtitle{And why should I use it?}

	A version control system tracks changes on projects.
	It is a useful tool to keep track of the following: \pause

	\begin{itemize}
		\item What changed since the last version? \pause
		\item When was the last modification? \pause
		\item Who was the last person on the team to modify this file? \pause
		\item Which version is the one I'm looking for? \pause
	\end{itemize}

	So you can avoid having filenames like \texttt{Projectfina3lastrev18.doc} or \texttt{Projectfinal(this\_is\_th\_one)}.

\end{frame}

% subsection what_is_git (end)

\subsection{Git is not Github} % (fold)
\label{sub:git-not-github}

\begin{frame}[allowframebreaks]
	\frametitle{Git is not Github}
	\framesubtitle{Let's get it right}

	\begin{figure}[h]
		\centering
		\includegraphics[width=0.45\textwidth]{logo}
		\label{fig:git_logo}
	\end{figure}

	\textbf{Git} is the software: a technology and a work flow methodology.

	\framebreak

	\begin{figure}[h]
		\centering
		\includegraphics[width=0.25\textwidth]{octocat}
		\label{fig:github_logo}
	\end{figure}

	\textbf{Github} is a website that offers `free' (as in free beer) hosting of open source projects. Github uses the Git methodology.
\end{frame}

% subsection git-not-github (end)
% section preamble (end)

% TODO:
% git workflow
% modify, add, commit, push
% collaborative tool
% branches, merges, tags
% how shell works
% actual git commands
% the help command
% more on the website
% thank you

\end{document}